
The mechanisms underlying how listeners' expectations are updated when a speaker is disfluent is only recently becoming better understood. 
   
   
   
   
Alongside influencing the interpretation of the literal message, comprehension biases with disfluencies also extend to the global pragmatic judgments made about an utterance, suggesting that the picture is more complex. Estimates of the speaker's metacognitive state are sensitive to delivery - utterances which are hesitant, disfluent, or with rising intonation are judged as less confident \citep{Brennan1995, Zuckerman1981}. This effect has been found in both offline measures of judgement - i.e. a post-hoc assessment of speaker's confidence \citep{Brennan1995, Swerts2005} - as well as during moment-to-moment processing \citep{Loy2016}.\\

\citet{Loy2016} looked at how listeners' pragmatic inferences about the speaker's honesty during online processing of speech was influenced by the presence of a disfluency. When presented with utterances concerning the location of a hidden reward, and asked to indicate whether the speaker was lying or telling the truth (by selection of one of two objects), participants' eye and mouse movements were found to be biased toward the referred-to object for fluent utterances, and the distractor object for disfluent utterances. As with the semantic effect of disfluencies on interpretation of an utterance \citep{Arnold2004,Arnold2007}, this pragmatic effect was found to emerge rapidly from the onset of critical words, and showed that disfluencies affect processes that are fundamental to language comprehension.\\

The majority of evidence for comprehension biases with disfluencies has investigated the influence of disfluency upon reference comprehension. Listeners' evaluation of the referent of expressions like "the bread" tends to begin within 200ms of the noun-onset \citep{Allopenna1998}. Given an utterance "... thee - um - bread", listeners pay more attention to both semantic \citet{Arnold2004} and pragmatic \citet{Loy2016} hypotheses about potential referents that are more likely given a preceding disfluency, complementing research which shows that listeners' reference comprehension is also sensitive to prosodic information \citep{Dahan2001} and discourse context \citet{Chambers2002}. The simplistic, extra-linguistic explanation for this sensitivity to the manner of an utterance's delivery is that paralinguistic cues simply heighten attention to certain aspects of the linguistic message, for instance a filled pause might simply heighten attention to the subsequent word \citep{FoxTree2001}. However, this fails to capture how disfluencies seemingly influence pragmatic judgments about a speaker's non-literal message, as in \citet{Loy2016}. 

 The low-level explanation of listeners' sensitivity to an utterance's delivery is that it is representative of a form of stochastic modelling of occurrences of disfluency. By this view, disfluency moderates listener's pragmatic inferences via a model of the distributions of disfluent and dishonest utterances, in which a disfluency/dishonesty link is present. Likewise, listeners' probabilistic expectations of upcoming words are updated according to the fact that disfluencies occur more often in reference to discourse-new and unfamiliar objects. Alternatively, but not incompatibly so, it is possible that listeners use the variation in delivery to draw inferences about the interaction between variation and meaning, and thus influence hypotheses about the thoughts and intentions of the speaker. This \textcolor{red}{speaker/situation}-modelling would view disfluency as driving inferential processing about the cause of disfluency - i.e production difficulty because of an unfamiliar/hard to describe object, or planning difficulty because of intention to deceive.\\

Examining how listeners' interpretion of disfluencies behaves in the presence of multiple competing explanations for speaker-disfluency can shed light on the cognitive processes underlying comprehension. If inferences are made about the most plausible cause of speaker-disfluency in a particular utterance, then one would expect to be able to extract multiple biases to disfluencies which are dependent on contextual information about the speaker. \citet{Arnold2007} showed that a bias towards unfamiliar, hard-to-describe objects following a disfluency is moderated by listeners' inferences about why a speaker delivered the utterance in that particular way. The results of both a gating task and an eye-tracking task showed that a) the discourse-new bias found in \citet{Arnold2004} could be extended to an unfamiliarity-bias, and b) this bias was reduced when the listener believed the speaker to have object-agnosia (inability to recognize objects). This shows that the effect which manner of delivery has on comprehension is dependent upon what is perceived to be the cause of the particular delivery, and is evidence of listeners' speaker-specific inferences affecting online comprehension - in this case the inference that both familiar and unfamiliar objects are equally difficult to describe for an object-agnosic speaker, affecting the listeners' online comprehension of disfluencies.\\

For listeners' perceptions about the cause of disfluency, belief that the speaker is object-agnosic provides an utterance-global explanation for speaker-disfluency. In this respect, whilst the integration of information evident in \citet{Arnold2007} clearly requires some form of inferencing, it is possible that listeners only engaged in the inferential process prior to speech onset. This could equate to essentially readjusting the statistical weight of the unfamiliarity-disfluency link, rather than repeatedly making inferential judgments about cause of disfluency as speech unfolds. The question remains whether listeners engage in inferential processing about the cause of disfluency \textit{during} online speech processing. \\

%Could also be worth working in Barr & Seyfeddinipur (2010) somewhere since we’re talking about systematic biases based on statistical distribution vs. something that is more speaker/situation specific, which is along the lines of what they look at?


The current study develops \citet{Loy2016}, and investigates whether the pragmatic deception-bias of disfluency is moderated by the presence of an utterance-local cause of speaker-disfluency - i.e. an explanation of disfluency which only presents itself \textit{during} the time-course of an utterance. The idea is that if utterance-local contextual information informs the online pragmatic judgments made about disfluency, this would suggest that listeners' inferential processing in attributing a cause to a disfluency is a) higly flexible, and b) occurs during the moment-to-moment processing of speech. For the experiment reported here, we develop the 'lie detection task' from \citet{Loy2016} in which listeners are asked to make judgments about whether a speaker is being deceitful or not in indicating the location of a reward ( "The treasure is behind the \textless referent\textgreater ".). In a visual world paradigm, listeners signalled their assessment of a speaker's honesty by clicking on one of two objects - interpreting an utterance as honest would result in a click on the referent, and as dishonest a click on the distractor. As well as manipulating the speaker's manner of delivery, utterances contained an alternative plausible cause of speaker-disfluency in the form of speaker-distraction. Utterances were presented to listeners under the guise of having been recorded outside in a busy street, and a strategically placed car-horn functioned as the speaker-distraction manipulation. Recording eye- and mouse- movements, we look at how the time course of pragmatic inferencing about speaker's disfluent message is influenced by the availability of an alternative, utterance-local cause of disfluency.

















%We analysed participants' responses to three post-test questions, assessing whether subjects themselves found the background noises distracting; whether they thought the noises might distract the speaker, and whether the noises may have caused the speaker's disfluencies. Table \ref{table:questions} shows these questions and the percentages of participants who gave affirmative responses (questions were open-ended). \\


