\documentclass[a4paper,man,natbib]{apa6}
%\usepackage[square]{natbib}
\usepackage{microtype}
\usepackage{mathtools}
\usepackage{hyperref}
\usepackage{stmaryrd}
\usepackage{tabularx}
\usepackage[normalem]{ulem}
%\linespread{1.3}
% use apa defaults
\hypersetup{hidelinks=true}
% let's not be garish

\newcommand*{\smex}[1]{\textit{#1}} % 'small example'
\newcommand*{\spex}[1]{``{#1}''} % 'spoken example'



\title{Contextual effects on online pragmatic inferences.}
\author{JK; JL; MC}
\date{\today}


\shorttitle{disfluency/distraction}
\abstract{Where the veracity of a statement is in question, listeners show a bias towards interpreting speaker disfluency as a sign of dishonesty. 
This bias is not limited to post-hoc judgements, but can also be found during online speech processing. 
The present study investigates whether listeners are influenced by contextual information about the potential causes of speaker disfluency. 
If listeners make inferences about the cause of a disfluency, then a plausible speaker-distraction may attenuate the inclination to interpret an utterance as dishonest.

Participants listened to a speaker describe the locations of treasure (``The treasure is behind [the]/[thee---uh---] \textless referent\textgreater .''), while viewing scenes comprising the referent and a distractor. 
They were told that not all utterances would be honest, and their task was to click on the suspected location of the treasure. 
In line with previous work, participants were more likely to click on the distractor when the description was disfluent, and this effect corresponded to an early fixation bias, demonstrating the online nature of the pragmatic judgement.

The present study, however, also manipulated the presence of a plausible external cause of speaker disfluency. 
To accomplish this, participants were told that all utterances were produced outside on a busy street, and all items were played over low-level ambient street noise.

When there was a plausible external cause of the speaker disfluency (a distracting noise, in the form of a car-horn), participants were more likely to initially fixate on the referent, only later fixating on and selecting the distractor. 
One account of these findings would suggest that participants made early inferences about the contextual causes of disfluencies, which were eventually overridden by a dishonesty bias for disfluent utterances.}


\begin{document}
\maketitle


\section{Background}
Everyday speech is for the most part spontaneous, and thus often disfluent, containing pauses, \spex{um}s, \spex{uh}s, repetitions, revisions, and mispronunciations.
Excluding silent pauses, naturally occurring speech has a rate of approximately~6 to~10 disfluencies per~100 words \citep{Bortfeld2001,FoxTree1995}.
The disfluent nature of speech is just one of many variable aspects of \emph{how} an utterance might be presented, and listeners must be able to cope with this variability in order to successfully understand a speaker.

Disfluencies in speech are not merely incidental.  Speakers are more disfluent when utterance planning involves low-frequency words \citep{Beattie1979}, less-preferred syntactic structures \citep{Cook2009}, discourse-new expressions \citep{arnold2000heaviness}, or a greater choice of expressive alternatives \citep{Schachter1991}.
%% get your citations right!  Arnold et al. (2003) is comprehension
In this way, disfluencies provide paralinguistic `cues' about the content of a speaker's message.
Research has shown that listeners can, and do, exploit these cues to make predictions about upcoming speech.
For example, following a disfluency, they are more likely to predict the introduction of a new object into the discourse, as indexed by visual world eye movements \citep{Arnold2004}, and less likely to have difficulty integrating an unpredictable word into its context, as shown by a reduction in the N400~ERP component \citep{Corley2007}.

Evidence from a series of eyetracking experiments suggests that predictions like these are sensitive to context.
\citet{Arnold2007} asked participants to click on depictions of easy-to-name (ice-cream) or harder-to-name (abstract symbol) items in response to auditory instructions.
Where the instructions were disfluent, participants were more likely to fixate harder-to-name items during an ambiguous colour adjective which preceded the item name.
Importantly, these fixation biases were modulated when participants were told that the speaker had object agnosia, and hence might be presumed to have difficulty naming easy-to-name items.
The fact that a prediction that a hard-to-name item will follow a disfluency can be modulated by contextual information suggests that on encountering a disfluency, participants are not merely making a stochastic prediction about what might be mentioned next.
Instead, they are actively modelling the speaker in order to account for the disfluency encountered and make situation-specific predictions.

However, the picture remains far less clear when the cause of the disfluency is local, in the sense that it could be assumed to be the cause of a specific instance of disfluency, rather than of a heightened probability of disfluency in general.
In \citeauthor{Arnold2007}'s Experiment~3, for example, local causes (beeps and construction noises, assumed to distract the speaker momentarily) did not affect the bias to fixate harder-to-name objects following disfluency.
Moreover, several studies have shown that listeners do not seem especially sensitive to the nature of the disfluency:  They have been shown to be affected by dog barks \citep{bailey2003disfluencies} and sine waves \citep{corley2011helps} when they are substituted for filled-pause disfluencies.
This sensitivity to non-linguistic interruptions sits poorly with the idea that the listener is modelling the speaker's production system, to anything greater than a superficial extent.

One reason that the picture may remain unclear is that, in the studies outlined above, the effects of disfluency are ephemeral.
Disfluency may affect what listeners think they are about to hear, but it has no lasting consequences at the message level:  The fluent and disfluent versions of the utterances used mean the `same thing'.
However, a parallel literature shows that disfluency may also have pragmatic effects.
%
%FO(A)K; other pragmatics; lying; Jia \ra{}question:  Now we're talking about \emph{pragmatic} effects, is there evidence of speaker modelling?
For example, \citet{Brennan1995} played listeners' answers to general knowledge questions which had been obtained during a production study.
The answers were digitally edited and were sometimes preceded by either a silent pause, or a filler.
Listeners in this study rated the answers as being less likely to be correct when the recorded answers were preceded by silence or fillers.
In this study, listeners' interpretations of a spoken message depend on the manner in which it is delivered \citep[see also][]{Swerts2005}.

In \citeauthor{Brennan1995}'s study, disfluency affects the `Feeling of Another's Knowing (FOAK)', presumably because it is perceived to be more effortful to retrieve answers of which one is uncertain, and therefore more likely to result in disfluency.
Another pragmatic effect which is often associated with effort is that of lying: \citet{Zuckerman1981} found hesitations in speech to be reliably associated with a perception of dishonesty, in both judgments made by speakers about themselves, and in judgments made about another speaker. 




% then pragmatics
% then immediate vs global context?
% then towards our design.



\bibliography{e2}

\end{document}
%%% Local Variables:
%%% mode: latex
%%% TeX-master: t
%%% End:
