\documentclass[a4paper,man,natbib]{apa6}
%\usepackage[square]{natbib}
\usepackage{microtype}
\usepackage{mathtools}
\usepackage{hyperref}
\usepackage{stmaryrd}
\usepackage{tabularx}
\usepackage[normalem]{ulem}
%\linespread{1.3}
% use apa defaults
\hypersetup{hidelinks=true}
% let's not be garish

\newcommand*{\smex}[1]{\textit{#1}} % 'small example'
\newcommand*{\spex}[1]{``{#1}''} % 'spoken example'
\newcommand*{\term}[1]{\emph{#1}} % introducing a new term
\newcommand*{\citegen}[1]{\citeauthor{#1}'s \citeyear{#1}}


\title{Contextual effects on online pragmatic inferences.}
\author{JK; JL; MC}
\date{\today}


\shorttitle{disfluency/distraction}
\abstract{Where the veracity of a statement is in question, listeners show a bias towards interpreting speaker disfluency as a sign of dishonesty. 
This bias is not limited to post-hoc judgements, but can also be found during online speech processing. 
The present study investigates whether listeners are influenced by contextual information about the potential causes of speaker disfluency. 
If listeners make inferences about the cause of a disfluency, then a plausible speaker-distraction may attenuate the inclination to interpret an utterance as dishonest.

Participants listened to a speaker describe the locations of treasure (``The treasure is behind [the]/[thee---uh---] \textless referent\textgreater .''), while viewing scenes comprising the referent and a distractor. 
They were told that not all utterances would be honest, and their task was to click on the suspected location of the treasure. 
In line with previous work, participants were more likely to click on the distractor when the description was disfluent, and this effect corresponded to an early fixation bias, demonstrating the online nature of the pragmatic judgement.

The present study, however, also manipulated the presence of a plausible external cause of speaker disfluency. 
To accomplish this, participants were told that all utterances were produced outside on a busy street, and all items were played over low-level ambient street noise.

When there was a plausible external cause of the speaker disfluency (a distracting noise, in the form of a car-horn), participants were more likely to initially fixate on the referent, only later fixating on and selecting the distractor. 
One account of these findings would suggest that participants made early inferences about the contextual causes of disfluencies, which were eventually overridden by a dishonesty bias for disfluent utterances.}


\begin{document}
\maketitle


\section{Background}
Everyday speech is for the most part spontaneous, and thus often disfluent, containing pauses, \spex{um}s, \spex{uh}s, repetitions, revisions, and mispronunciations.
Excluding silent pauses, naturally occurring speech has a rate of approximately~6 to~10 disfluencies per~100 words \citep{Bortfeld2001,FoxTree1995}.
The disfluent nature of speech is just one of many variable aspects of \emph{how} an utterance might be presented, and listeners must be able to cope with this variability in order to successfully understand a speaker.

Disfluencies in speech are not merely incidental.  Speakers are more disfluent when utterance planning involves low-frequency words \citep{Beattie1979}, less-preferred syntactic structures \citep{Cook2009}, discourse-new expressions \citep{arnold2000heaviness}, or a greater choice of expressive alternatives \citep{Schachter1991}.
In this way, disfluencies provide paralinguistic `cues' about the content of a speaker's message.
Research has shown that listeners can, and do, exploit these cues to make predictions about upcoming speech.
For example, following a disfluency, they are more likely to predict the introduction of a new object into the discourse, as shown by visual world eye movements \citep{Arnold2004}, and less likely to have difficulty integrating an unpredictable word into its context, as indexed by a reduction in the N400~ERP component \citep{Corley2007}.

Evidence from a series of eyetracking experiments suggests that predictions like these are sensitive to context.
\citet{Arnold2007} asked participants to click on depictions of easy-to-name (ice-cream) or harder-to-name (abstract symbol) items in response to auditory instructions.
Where the instructions were disfluent, participants were more likely to fixate harder-to-name items before they heard the item name.
Importantly, these fixation biases were modulated when participants were told that the speaker had object agnosia, and hence might be presumed to have difficulty naming easy-to-name items.
The fact that a prediction that a hard-to-name item will follow a disfluency can be modulated by contextual information suggests that on encountering a disfluency, participants are not merely making a stochastic prediction about what might be mentioned next.
Instead, they are actively modelling the speaker in order to account for the disfluency encountered and make situation-specific predictions.

However, the picture remains far less clear when the cause of the disfluency is local, in the sense that it could be assumed to be the cause of a specific instance of disfluency, rather than of a heightened probability of disfluency in general.
In \citeauthor{Arnold2007}'s Experiment~3, for example, local causes (beeps and construction noises, assumed to distract the speaker momentarily) did not affect the bias to fixate harder-to-name objects following disfluency.
Moreover, several studies have shown that listeners do not seem especially sensitive to the nature of the disfluency:  They have been shown to be affected by dog barks \citep{bailey2003disfluencies} and sine waves \citep{corley2011helps} when they are substituted for filled-pause disfluencies.
This sensitivity to non-linguistic interruptions sits poorly with the idea that the listener is modelling the speaker's production system, to anything greater than a superficial extent.

One reason that the picture may remain unclear is that, in the studies outlined above, the effects of disfluency are ephemeral.
Disfluency may affect what listeners think they are about to hear, but it has no lasting consequences at the message level:  The fluent and disfluent versions of the utterances used mean the `same thing'.
However, a parallel literature shows that disfluency may also have pragmatic effects.
%
%FO(A)K; other pragmatics; lying; Jia \ra{}question:  Now we're talking about \emph{pragmatic} effects, is there evidence of speaker modelling?
For example, participants in \citegen{Brennan1995} study were played recordings of answers to general knowledge questions which had been obtained during a production study.
The answers were digitally edited and were sometimes preceded by either a silent pause, or a filler.
In line with the suggestion that speakers use disfluency to manage difficulty in retrieving information \citep{Smith1993}, participants rated the answers as being less likely to be correct when the recorded answers were preceded by silence or fillers:
In other words, their interpretations, rather than simply predictions, of the utterances they heard were directly affected by disfluency \citep[see also][]{Swerts2005}.
Listeners faced with disfluency had less confidence in the speaker's knowledge (a weaker ``Feeling of Another's Knowing'', or FOAK), and therefore in the factual correctness of what was being said.

%%% NB I couldn't find much evidence of "other pragmatic" effects.  I
%%% do seem to recall some social psych papers talking about
%%% disfluency as hedging; would be great if we could find those

As well as producing language about which they have little confidence, speakers can easily produce language which they know to be false.
This form of lying \citep[see][for another]{frankfurt05} is often associated with cognitive effort.
According to this view, the increased load involved in formulating and uttering a lie may lead speakers to provide verbal and non-verbal cues to deception, including disfluency \citep{Zuckerman1981,depaulo2003cues}.
% note careful phrasing above so as not to contradict Jia :)
Listeners' interpretations appear to reflect such an idea: \citet{Zuckerman1981} found hesitations in speech to be reliably associated with a perception of dishonesty, in both judgments made by speakers about themselves, and judgments made about another speaker.
However, the accuracy of disfluency as a cue to deception is questionable: the literature surrounding deception and production of disfluency is not clear, with some evidence suggesting a converse account: that disfluency occurs more frequently during truth-telling than during deception \citet{Arciuli2010markers, Arciuli2009lies, Benus2006pauses}.
\citet{DePaulo1982actual} is one study which shows this potential mismatch between disfluency as an actual and as a perceived cue to deception:
the rate of filled pauses produced by speakers did not differ during descriptions they made about people whom they liked/disliked from descriptions made when they were asked to pretend to feel the opposite way about them. 
%rephrase above explanation. not clear.
However, when listening to the descriptions made by other participants, higher rates of filled pauses were associated with an interpretation that the speaker was being deceitful.


%%%%%%%% need link here. importance is comp not produ.
In both FOAK and lying research, the proposed mechanism by which the interpretation of what is said is affected by disfluency is via \term{speaker modelling}:
By reverse inference, disfluency is a symptom of cognitive difficulty, and cognitive difficulty is the consequence of limited knowledge (FOAK) or of inventing a situation (lying).
To conclude that the speaker is lying requires reasoning about his or her cognitive state.
However, listeners may in fact not reason in this way:
Instead, they may heuristically associate certain aspects of spoken performance with lying.


%%% this para needs some TLC
One reason for believing that the association between disfluency and lying is heuristically calculated is evidence from \citet{Loy2016}, which highlights the speed at which pragmatic interpretations are made.
\citeauthor{Loy2016}'s study was framed as a treasure hunting game, in which listeners were asked to guess the location of a reward by clicking on one of two depicted objects in each trial.
Participants heard recorded utterances which indicated the location of the treasure, and which were either fluent or disfluent (\spex{The treasure is behind [the]/[thee---uh---] \textless referent\textgreater}).
Participants were told that the speaker would be dishonest half of the time. 
The judgements which participants made about the speaker's honesty in each trial were implictly measured by examining which of the two objects they clicked:
Clicking on the named object corresponded to a judgement that the speaker was telling the truth, whereas a click on the other object meant that the speaker was thought to be lying.
In line with previous research \citep{Zuckerman1981},  participants were less likely to click on the named object following disfluent utterances (and instead, tended to click on the object which had not been mentioned).
Importantly, eye- and mouse-tracking records showed that this effect emerged as soon as it became clear which of the two objects was being named:
In other words, participants' pragmatic judgements were shown to be influenced by disfluency at the earliest detectable moment.
If speaker modelling is occurring, any inferences regarding the cause of a given disfluency would have to have been made very fast.


Moreover, the evidence suggests that, in cases where disfluency might modulate listeners' predictions, contextual inferences are limited to more global factors which would affect the speaker throughout a discourse \citep{Arnold2007}.
%% the last sentence above doesn't quite fit here for me.. 
% or maybe just needs an extra sentence?
This leaves open the possibility that the processing of disfluent speech results from an heuristic association between disfluency and lying, as a consequence of the (potentially inaccurate) belief that speakers are disfluent when they lie. 
% I don't like the last 3 sentences. needs rewriting. It's really a fairly important segue into the idea though... mention Zuckerman1981 people believe themselves to be more disfluent when they lie?
%this is sort of 'superficial speaker-modelling' again..
Thus in \citeauthor{Loy2016}'s study, subjects may have engaged in a superficial form of speaker modelling inasmuch they may have made an inference about speakers' production systems in general.
Their actual processing of a specific disfluency, however, need not have involved any algorithmic reasoning about its cause.
%don't like 'algorithmic reasoning', but no matter 


In real human interactions, however, the automatic association of disfluency with dishonesty would be counterproductive, not least because disfluency has been shown to index (at least) uncertainty as well as lying, there would at least have to be some heuristic to determine which interpretation was sensible.
The current study aims to better understand the underlying mechanisms involved in processing disfluent speech: do listeners rely on rule-of-thumb stochastic associations, or do they engage in \term{active} speaker modelling and reason about the cause of a specific disfluency? 
% don't like the word 'reason' there...
To investigate this, we look at whether the pragmatic judgments listeners make about a speaker's honesty are modulated by a local cause of disfluency. 
We develop \citegen{Loy2016} treasure hunting game, with the addition of manipulating the presence of a disruptive (and so plausibly distracting) noise in the speaker's environment, providing an alternative explanation for speaker disfluency.
Thus the study followed a 2 (fluent/disfluent) x 2 (distraction absent/present) design.
Subjects participated in the same visual world paradigm game in which they guessed the location of some treasure based on utterances made by a potentially deceitful speaker.
In this experiment, however, utterances were presented to listeners under the guise of having been recorded outside in a busy street, and low-level ambient noise was present behind every utterance.
This provided the context for the potentially distracting noise, which took the form of a car-horn. 
As in \citet{Loy2016}, we look at eye- and mouse- movements, to study the time course of listeners' pragmatic judgments about the honesty of an utterance, as well as their final interpretation (object-clicked).
If the association between disfluency and lying is heuristically calculated, then the within-utterance occurrence of an alternative explanation for a disfluency (the car-horn) should not influence listeners' responses to the disfluency.
Alternatively, if listeners are engaging in an active speaker modelling to account for a given disfluency, we would expect the association between disfluency and lying to be modulated by the presence of the car-horn. 


\bibliography{e2}

\end{document}
%%% Local Variables:
%%% mode: latex
%%% TeX-master: t
%%% End:
